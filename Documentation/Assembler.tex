 
\documentclass{article}
\usepackage[bottom]{footmisc}
\newcommand{\V}{\verb}
\newcommand{\x}{$\textbf{X}$}
\newcommand{\y}{$\textbf{Y}$}
\newcommand{\s}{$\textbf{S}$}
\newcommand{\A}{$\textbf{A}$}
\newcommand{\mx}{$\textbf{M}_{\textbf{X}}$}
\newcommand{\my}{$\textbf{M}_{\textbf{Y}}$}
\newcommand{\ma}{$\textbf{M}_{\textbf{A}}$}
\newcommand{\q}{$\textbf{Q}_{\textbf{1}}$}
\newcommand{\qq}{$\textbf{Q}_{\textbf{2}}$}
\newcommand{\pc}{$\textbf{PC}$}
\newcommand{\J}{$\textbf{J}$}



\begin{document}
\title{Assembler documentation}

\section{General}
The VM features a very basic assembler capable of little more than adress
resolution. But yet it gives us an ability to create some neat little programs.
\section{Syntax}
The syntax for the assembly code is pretty straight forward. Each declaration is
written on a single line. There are a few reserved identifiers:

\begin{tabular}{|c | c| c |}
\hline
Indentifier & Name & Description\\
\hline
\V+%<text>+ & Comment  & Will be ignored by the assembler\\
\hline
\V+#<name>+ & Label & Declares a label called \V+<name>+ \\
\hline
\V+@<name>+ & Value & Declares a value called \V+<name>+\\
\hline
\V+:<data>+ & Raw input & Returns \V+<data>+ as is \\
\hline
\verb+$<a>+ & Address & Derefferences \V+a+\\
\hline
\V+x+ & x & The \x \ register \\
\hline
\V+y+ & y & The \y \ register \\
\hline
\V+s+ & s & The stack \s \\
\hline
\V+q1+ & IRQ1 & The \q \ register\\
\hline
\V+q2+ & IRQ2 & The \qq \ register\\
\hline
\end{tabular}
\\
\\
The names given to \emph{label}s and \emph{value}s can contain any characters except for
whitespace ones.\\
Operations are declared in a straightforward approach as:\\
\V+<opcode> <arg1> <arg2>+ \\
Wich arguments are allowed are dependent upon the opcode.\\
Any non whitespace character can be used for names of labels and values. 

\section{Inner Workings}
The assembler works in a fairly straight forward way. The first step in the
process of asemblying is the lexiographical analysis in wich the lines in the
text file gets tokenized. In this tage an ``intermediate structure''
\footnote{The use of the word structure here is a bit ambigous since it
actually is a structure in sml. But in this text it will reffere to an abstract
structure of data.} gets constructed.
This is an object wich contains all of the labels, values and alist of the tokenized lines. The list of 
the tokens contains tupels of
\verb+(label,offsett,token)+ wherer the lable is the last initialized label and
the offset is how many addresses away from that line the current token is. Allof
the labels and values will not be assigned an adress in this phase. It is in
theis phase where the opcodes and their arguments gets converted in to there
corresponding numerical instruction code. It is also during this phase in wich
the syntax gets checked. If a syntax error is encountered the assembler will
stop emideatly. When the lexiographical analysis has been completed a check for
duplicate pointer declaration is performed.

The next phase in the assembly is the address resolution phase. This is done in
two phases, in the first one the labels gets resolved and in the second the
values gets  resolved. It begins by first resolving the labels.
This is done by first giving the assembler a \enph{base address} wich is the adress of the first label. As of now the first
non comment line in the input file has to be a label since every line has to
have a label assigned to it. Then the address resolution function continues down
the intermediate sturcture and remebering wich line it is at and what its last
read label was. When it runns into a new label token it will sett the new label
to its current address and then continue on untill it has gone trough the entire
intermediate structure. After the labels have been resolved the assembler starts
to resolve the values. This is done in a very straightforward way. The assembler
just looks at the last address of the last entry in the output of the firsst
pass and looks at the last address, adds one to it and the just places the all
the values in after that address in the same order as they appeared in the file.
After all the adress has been resolved the assembler runns trough the list of
tokens and replaces every pointer token with its correct adress.

After this is completed the assembler finalizes the code by converting
everything into a list of integers wich then gets outputed to a file. And that
children is how assembly code gets turned in to machine code.

\section{Usage}
\subsection{General}
One noteworthy thing to point out is the limitations on the arguments. Due to
limitations in the VM only one ``none registry'' argument can be used for any
operation. A ``non registry'' argument is one wich is either a number or a
a pointer. Derefferencing a pointer is a registry operations so they are valid.
Below follows some examples:
\begin{verbatim}
#label
@value
MOV label value    This is not accepted
MOV $label $value  This is perfectly fine
MOV 10 value       This is invalid
MOV 10 $value      But this is
ADD 1 10           This is invalid
ADD $1 $10         This is valid
ADD 1 $1           So is this
\end{verbatim}

\subsection{Registers}
The useage of the registers is pretty straight forward. One has to remeber that
\V+q1+ and \V+q2+ are write only registers and that \V+s+ cant be used for
addressing so \verb+$s+ is not allowed and will generate a \verb+syntax+ error.
It is also good to keep in mind that all operations reading from the stack will
consume what is on top of the stack.

\subsection{Pointers}
Using pointers is fairly straight forward. Alltough one has to keep in mind how
the addresses are resolved. All pointers will be resolved after the
tokenazation fo the code. First the \emph{labels} will be resolved and then the
\emph{values}.
This means that the first \emph{value} will lie after the last line of code. Since the
address of a \emph{label} depends on where in the code their addresses are easy to
reason about. However for \emph{values} things are bit different. Since vlues
will be given addresses wich are ``independent'' of where in the code they appear it is
hard to reason about the address of a \emph{value}. Although the \emph{value} pointers are
resolved in order the first \emph{value} declared whill lie intermediatley after the
last line of code and the last \emph{value} declared will lie ``at the end'' of the
memmory used by the program. This can be exploited to use rellative addressing.
Allthoug great care has to be taken.

Its important to remember that all pointers are reffered to troughout the entire
program therefor it's not allowed to define two pointers with the same name. If
this where to be allowed it would generate unpredictable behaviour so instead
the assembler will return a \V+assembler+ error.

\emph{label}s and \emph{value}s are interchangable. Since opcodes takes pointers as arguments
and has no idea weather or not they are \emph{label}s or \emph{value}s. From this the need for
caution arises. Since one can use \emph{value} pointers as arguments to jump operation
like this:
\begin{verbatim}
@bad_idea
ADD x y
MOV s x
MUL x y
JMP bad_idea
\end{verbatim}
Since it is not known what where \V+bad_idea+ points jumping to it is suicidal.

Since pointers are just numbers under the hood one needs to take into account
weather or not one uses them for their adress orr for their \emph{value}s. Here is some
examples
\begin{verbatim}
@pointer
% This stores x in pointer
MOV x value
% This stroes x in the address wich is
% stored at pointer
MOV x $value
% This adds one to the value stored at 
% pointer
ADD $pointer 1
% This adds one to the address of pointer
ADD pointer 1
\end{verbatim}

Pointers are imutable and once they has been declared they can not be changed.
One has to do some tricking to achive relative addressing using \emph{labels} or
\emph{values}.


\subsubsection{Labels}
\emph{Labels} are declared using the \V+#+ identifier.
\emph{Labels} are resolved first and their addresses
correspond to location in the code where they are written. For example:
\begin{verbatim}
MOV x y
#loop
INC x
MOV x s
JMP loop
\end{verbatim}
In this code \V+loop+ points to the address where \V+INC x+ is stored. In the
tokenization of the assembly code the lines where a pointer is defined will be
ignored and the address where the next instruction or raw entry occurs. This can
lead to that poorly written code becomes ambigous. For example:
\begin{verbatim}
MOV x y
#loop
#silly
INC y
\end{verbatim}
Here \V+loop+ and \V+silly+ will both point to the same address wich is silly.

Because tokenaization of the code happens before the address resolving a \emph{label}
will be ``in scope'' troughout the enitre code. So this code is perfectly valid:

\begin{verbatim}
MOV x y
JMP ahead
INC x
ADD x y
#ahead
ADD s x
\end{verbatim}
The \V+JMP ahead+ will jump to \V+ADD s x+ even though the \V+ahead+ flag is
defined after the jump. This was not a concious design choise but it is actually
quite usefull since one can define subroutines anywhere in the code wich can be
accesed form anywhere in the code.\\
One possible piffall arises due to the fact that the assembler does not know the
difference between a \emph{label} and a \emph{value} after their adresses has been resolved.
So this code is valid assembly code:
\begin{verbatim}
#loop
ADD s x
MOV s $loop
JMP loop
\end{verbatim}
Allthough what this will do is that it will change what is at the address of
\V+loop+. But there \V+ADD s x+ lies! This is what is known as self-modifiying
code and it's the spawn of satan and should be avoided like one avoids Miami
beach during spring break. Allthough in some cases the interchangability of
\emph{value} and \emph{label} can be verry usefull if one wants to have ``arrays'' in ones
code. This is easily achived like this:
\begin{verbatim}
#array
:0
:1
:2
:3
\end{verbatim}
Here \V+array+ can be used as a pointer to the array. One can then manipulate
the array trough using rellative arressing of \V+array+ like this:
\begin{verbatim}
ADD 2 array
MOV s y
MOV 5 $y
#array
:0
:1
:2
:3
\end{verbatim}
This code would change the 2 into a 5. But great care needs to be taken since
one could easily end up outside of the ``array'' and corrupt the program.

\subsubsection{Values}
\emph{Value}s are far more straight forward than \emph{label}. One only has to
take into account that what address a \emph{value} is given is somewhat
independet of where in the code it gets defined.

\subsection{Jumping}
Doing oridary jumps using the \V+JMP+ operatioon is very straight forward. The
machine will just jump to the address given to the \V+JMP+ operator.

But for conditional branching things become a little bit less obvious. If the
test given to a conditional test fails the machine will skip the next
instruction. Letts illustrate this with a few examples:
\begin{verbatim}
MOV 10 x
BLE x 2
JMP this_does_not_happen
BGR x 2
JMP this_happens
\end{verbatim}


\end{document}
