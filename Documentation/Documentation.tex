\documentclass{article}
%\usepackage[bottom]{footmisc}
\newcommand{\V}{\verb}
\newcommand{\x}{$\textbf{X}$}
\newcommand{\y}{$\textbf{Y}$}
\newcommand{\s}{$\textbf{S}$}
\newcommand{\A}{$\textbf{A}$}
\newcommand{\mx}{$\textbf{M}_{\textbf{X}}$}
\newcommand{\my}{$\textbf{M}_{\textbf{Y}}$}
\newcommand{\ma}{$\textbf{M}_{\textbf{A}}$}
\newcommand{\q}{$\textbf{Q}_{\textbf{1}}$}
\newcommand{\qq}{$\textbf{Q}_{\textbf{2}}$}
\newcommand{\pc}{$\textbf{PC}$}    
\newcommand{\J}{$\textbf{J}$}
\title{FML a VM implemented in SML}
\author{Henrik Sommerland, Oskar Ahlberg, Aleksander Lunqvist}
\date

\begin{document}
\maketitle
\section{Introduction}
For our project we have decided to build a virtual machine(VM) in SML. The name
FML is just an arbitrary thre letter name and has no meaning or interpertation.
The VM is a RISC machine using a Von-Neuman architecture. It has a very
minimalistic instruction set. The design of FML resembles those of older 8-bit
architectures such as the MOS 6510 and the Z80 microprocessors commonly in use
during the late 70s and early 80s. The FML machine has no ``bus width'' and
works exclusivley with signed integers\footnote{The details of the integers
used are dependent on which SML implementation is used}. The lack of a physical
bus enables the VM to do things wich an ordinary CPU could not achive such as
reading from two registers at the same time. Even though the cpu have very few
operations (only 27) a very effective instruction set architecture
makes these operations very flexibels and there are roughly 600 accepted 
instructions.

So even though FML is a very minimalistic machine it is quite powerfull.

We have allso built a fully featured assembler for the FML machine.

\end{document}
