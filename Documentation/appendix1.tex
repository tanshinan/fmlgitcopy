\documentclass[a4paper]{article}
\usepackage[english]{babel}
\usepackage[utf8]{inputenc}
\usepackage{amsmath}
\usepackage{graphicx}
\usepackage[colorinlistoftodos]{todonotes}
\usepackage[normalem]{ulem}
\usepackage[T1]{fontenc}

\title{Appendix x.1}
\author{Henrik Sommerland, Oskar Ahlberg, Aleksander Lunqvist}
\begin{document}
\maketitle

\begin{abstract}
In this file we will go thru the functions of the Components.sml, and explain the functions that we are using.
\end{abstract}
\tableofcontents
\newpage
\section{Introduction}

The following structures and signatures are present in Components are the Ram, Stack, Register and ProgramCounter.
The structure can be seen as a new part of the library in SML, we will now go thru the structures and the encapsulated functions.

\section{Components.sml}

\subsection{\uline{The Ram structure}}
\subsubsection{Synopsis}
\texttt{signature RAM\\
structure Ram :\textgreater RAM}\\

The Ram structure provides a base of the functions of a ram memory that is a interglacial part of the virtual machine.

\subsubsection{\uline{INTERFACE}}


   \texttt{type memory = int array
	\\val initialize : int $\rightarrow$ memory 							
 	\\val getSize : (memory) $\rightarrow$ int 
    \\val write :(memory * int * int ) $\rightarrow$ memory	  				
	\\val read : (memory * int) $\rightarrow$ int							
	\\val load : (memory* int list) $\rightarrow$ memory					
	\\val writeChunk : (memory * int * (int array)) $\rightarrow$ memory
	\\val readChunk : (memory * int * int) $\rightarrow$ int array			
	\\val dump : memory $\rightarrow$ string}
    \\
    \\
    \\
\\
\\
\subsubsection{\uline{Description}}

\texttt{val initialize : int $\rightarrow$ memory} \\	
	Initialize the ram to a memory with the size of int, when int > 0
\\
\texttt{val getSize : (memory) $\rightarrow$ int}\\
		Gets the size of the ram (memory) 
\\
\texttt{val write :(memory * int * int ) $\rightarrow$ memory}\\
		write takes a memory and writs a new value of int at the pointer of the first int and returns the memory
\\
\texttt{val read: (memory * int) $\rightarrow$ memory}\\
		read takes a memory and reads the value of the place of int
\\
\texttt{val load: (memory * int list) $\rightarrow$ memory}\\
		load takes a list of values and loads them to the memory
\\
\texttt{val writeChunk: (memory* int *( int array)) $\rightarrow$ memory} \\
		writeChuck takes a memory and a start pointer and adds a chunk to the memory
\\
\texttt{val readChunk: (memory * int *int) $\rightarrow$ int array} \\
		readChumk takes a memory and reads a chunk form first int to the last int and gives the values as an int array
\\
\texttt{val dump: memory $\rightarrow$ string}\\
		dump takes a memory and returns the value as strings
\\
\\
This concludes the RAM structure
\\
\\
\subsection{\uline{The Stack structure}}
\subsubsection{Synopsis}
\texttt{signature STACK \\
structure Stack :\textgreater STACK}
\\
\\
The Stack structure provides a base for the stack part of the Pc structure.

\subsubsection{\uline{INTERFACE}}
	\texttt{datatype stack = Stack of (int list)\\	
	val empty : stack \\
	val push : stack * int $\rightarrow$ stack\\					
	val pop : stack $\rightarrow$ stack\\						
	val top : stack $\rightarrow$ int\\						
	val isEmpty : stack $\rightarrow$ bool \\						
	val dumpStack : stack $\rightarrow$ string}\\					 


\subsubsection{\uline{Description}}

	\texttt{val empty : stack}\\
		is a definition of a empty Stack
\\
	\texttt{val push : stack * int $\rightarrow$ stack}\\					
		takes a stack and adds the value of int to the stack.
\\
	\texttt{val pop : stack $\rightarrow$ stack}\\					
		takes a Stack and �pop�s the first element of the stack.
\\
	\texttt{val top : stack $\rightarrow$ int}\\							
		takes the stack and returns the first element of the stack
\\
	\texttt{val isEmpty : stack $\rightarrow$ bool}\\						
		takes a stack and checks if it is empty if it is then true else false.
\\
	\texttt{val dumpStack : stack $\rightarrow$ string}\\
		takes a stack, then pops the stack until its empty and returns all values as string
\\
\\
This concludes the stack structure.
\\
\\
\subsection{\uline{The Register structure}}
\subsubsection{Synopsis}

\texttt{signature REGISTER\\
structure Register :\textgreater REGISTER}\\
\\
The Register structure provides a base structure of the different register that is contained in the Pc as well the Virtual machine. The vm has two different registers.\\
\subsubsection{\uline{INTERFACE}}
	\texttt{datatype reg = Reg of int 
\\	
	val setData : (reg * int) $\rightarrow$ reg 				
\\	val getData : reg $\rightarrow$ int						
\\	val increment : reg $\rightarrow$ reg						
\\	val decrement : reg $\rightarrow$ reg						
\\	val dumpRegister : reg $\rightarrow$ string}
\\
\\
\subsubsection{\uline{Description}}

	\texttt{val setData : (reg * int) $\rightarrow$ reg}\\
		Setups a new Register \\
	\texttt{val getData : reg $\rightarrow$ int}\\
		Gets the value of the reg as an int\\					
	\texttt{val increment : reg $\rightarrow$ reg}\\		
		Takes a reg and increment it with one.\\	
	\texttt{val decrement : reg $\rightarrow$ reg}\\					
		Takes a reg and decrements it with one.\\
	\texttt{val dumpRegister : reg $\rightarrow$ string}\\
		Takes the register and adds all elements to a string.\\
\\
This concludes the Register structure.
\\
\\
\subsection{\uline{The Program\textunderscore Counter structure}}
\subsubsection{Synopsis}
\texttt{signature PROGRAM\textunderscore COUNTER\\
structure ProgramCounter :\textgreater PROGRAM\textunderscore COUNTER}\\
\\
The ProgramCounter structure provides the foundation of the Virtual machine this is the hearth of the 
\\
\subsubsection{\uline{INTERFACE}}
	\texttt{datatype pc = Pc of (int * Stack.stack * Register.reg * Register.reg)
\\
	val incrementPointer : (pc * int) $\rightarrow$ pc\\
	val jump : (pc * int) $\rightarrow$ pc\\
	val subroutineJump : (pc * int) $\rightarrow$ pc\\ 						
	val return : pc $\rightarrow$ pc\\	
    val interrupt : (pc * int) $\rightarrow$ pc\\
    val dumpPc : pc $\rightarrow$ string}\\
\\
\subsubsection{\uline{Description}}
	\texttt{val incrementPointer : (pc * int) $\rightarrow$ pc}\\
		Takes a Pc and adds a int > 0\\							
	\texttt{val jump : (pc * int) $\rightarrow$ pc}\\
		Takes a Pc and jumps the pc counter to the value of int > 0\\
	\texttt{val subroutineJump : (pc * int) $\rightarrow$ pc}\\
		Takes a Pc and preforms SubrutineJump with the value of  int > 0 and adds the  value of the pointer + 1 to the stack\\						
	\texttt{val return : pc $\rightarrow$ pc}\\
    	Takes a pc and gets the value from the pointer and pops the stack with the value\\
	\texttt{val interrupt : (pc * int) $\rightarrow$ pc}\\							
		if the value of a is 1 or 2, then the value of i is added to s\\
	\texttt{val dumpPc : pc $\rightarrow$ string}\\
		Takes a pc and dumps the content of the pc as a string (the Pc contained a pointer, Stack, and tow registers)\\
\\
This concludes the ProgramCounter structure
\\
\end{document}