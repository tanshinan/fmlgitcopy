\documentclass{article}
\usepackage{amsmath}
\usepackage{listings}  
\usepackage{verbatim}
\newcommand{\x}{$\textbf{X}$}
\newcommand{\y}{$\textbf{Y}$}
\newcommand{\s}{$\textbf{S}$}
\newcommand{\A}{$\textbf{A}$}
\newcommand{\mx}{$\textbf{M}_{\textbf{X}}$}
\newcommand{\my}{$\textbf{M}_{\textbf{Y}}$}
\newcommand{\ma}{$\textbf{M}_{\textbf{A}}$}
\newcommand{\q}{$\textbf{Q}_{\textbf{1}}$}
\newcommand{\qq}{$\textbf{Q}_{\textbf{2}}$}
\newcommand{\pc}{$\textbf{PC}$}
\newcommand{\J}{$\textbf{J}$}
\newcommand{\V}{\verb}

\title{Specs for a lame VM implemented in SML}


\begin{document}
\maketitle
\\
\large{\textbf{
THIS DOCUMENT IS CLOSED AS OF 21/2 2014
NO MAJOR CHANGES ARE ALLOWED. ONLY MINOR CHANGES SUCH AS 
TYPESETTING AND SPELLING ARE ALLOWED UNLESS A VERY GOOD
REASON IS PRESENTED
}
}\\
\\
\normalsize
This VM is a very simple and minimalistic machine. It has an memory of
arbitrary size which is specified at run-time. All data in the VM is signed
integers. The VM is asynchronous and each operation takes one step.
\section{Structure}
The VM consists of 9 components. Two general purpouse registers(\x \ , \y ), one
general purpose stack (\s), One virtual read only register (\A), One jump stack
\J, Two IRQ adress registers (\q \ , \qq), One ``ALU'', One program counter
(\pc) and of course a random acces memory.

\subsection{The general purpose registers}
The two general purpose registers \x \ and \y \ are both capable of being used
for all arithmetic operations and their values can also be used as addresses.
These two registers can be incremented and decremented.

\subsection{The stack}
The stack \s \ is a standard LIFO stack of unlimited size. The stack can not be
used for adressing. One can not read the top of the stack without popping it. If
one tries to get a value from an empty stack an exception will be raised and the
VM must halt.

\subsection{The Argument Register}
Now this is just a virtual read only register. The argument \A \  is only
accesible if the instruction being executed takes a predefined argument. The
argument will be the value of the memory location after the location at which
the \pc \  is currently pointing. No well formed instruction should refere to
\A \ unless it is supposed to.

\subsection{The Jump Stack}
The jump stack \J \ is not accesible by anything besides the \pc. The program
counter is of infinite size. The jump stack is responsible for keeping track
of the return address when a subroutine is performed. Every time someone
issues a subroutine jump the current address will be pushed onto the stack.
When a return jump is issued \J \ gets popped and its value gets assigned to
the \pc. The top entry on the stack can not be accsessed without popping the
stack.
If someone tries to execute a return jump if the jump stack is empty a exception 
shall be raised and the VM must crash.

\subsection{The IRQ registers}
The IRQ registers \q \ and \qq \ are two pointers to the memory. These are two
write only registers and can only be read by the \pc. The IRQ registers can be
assigned values like all the other registers.  If a interrupt is issued the \pc
\ will be assigned to the value of the corresponding IRQ register and the current
value of the \pc \ will be pushed onto \J.

\subsection{RAM}
The RAM in this machine works pretty much like any other random access
memory.\\
The RAM is devided into three part. There are some special cells with negative
addresses. These addresses are read only and contain information regarding the
layout of the memory. Any write to these special locations will result in the
VM crashing. The main part of the RAM is then sectioned in two two parts. One
``ordinary'' part and one external part. Prepherial components such as I/O
handlers and such can only accsess the external memory.\\
The two special addresses are -1 where the size of the entire memory is stored
and -2 stores the address at which the external memory starts.

\section{ISA}
Every opcode is represented by a integer where each digit provides
information about what the VM is to do in that step. The digits are from right
to left as follows.

\begin{description}
  \item[First] Read location
  \item[Second] Write location
  \item[Third] Arithmetic operations
  \item[Fourth] Logic operations
  \item[Fifth] Jump operations
  \item[Sixth] Special
\end{description}
Below is a table describing what each digit value corresponds to:
\begin{center}
\begin{tabular}{l || *{10}{c |}}
Value & 0 & 1 & 2 & 3 & 4 & 5 & 6 & 7 & 8 & 9 \\
\hline
Read & \x & \y & \s  &\mx & \my & \ma & \A & & & \\

Write & \x & \y & \s  &\mx & \my & \ma & \q& \qq & \A &\\

Arit &  & $++$ & $--$ & $+$ & $-$ & $\star$ & $\div$ & \%  &  & \\

Logic &  &  $=$ & $<$ & $>$ & $<<$ & $>>$ & and & or & xor & not\\

Jump & & J & B_{=} & B_{<} & B_{>} & J_{SR} & Ret & & & \\

Special & & H & Se & \V+POP+ &  &  &  &  &  &  \\	
\end{tabular}
\end{center}
Here \mx , \my and \ma is to be read as address of \x , \y \ and \A.  All
arithmetic and logic operations operations writes their output to the stack.\\
Here some examples follows:\\
\verb+000042+ $\rightarrow$ Move value at \s \ to memory cell at the address
stored in \y. \\
\verb+000401+ $\rightarrow$ Get \x \ mod \y \ and write result to \s \\
\verb+020046+ $\rightarrow$ Skip next instruction if \my \ is equal to \A \\
\newpage

First i would like to  mention that
\verb+000000+ will be the \verb+NOP+ operation since it would translate to just
moving \x \ to \x \ . 
We can now group the instructions in to numerical ranges:

\begin{tabular}{l l}
  \V+000000+ & \V+NOP+ \\
  \hline
  \V+000001-000076+ & Move operations \\
  \hline
  \V+000100-000776+ & Arithmetic operations \\
  \hline
  \V+001000-009076+ & Logic operations \\
  \hline
  \V+010000-070000+ & Jump operations \\
  \hline
  \V+100000+ & Special \\
\end{tabular} \\
As is apparent from this list many values would yield invalid or
nonsense operations.
The instruction decoder must take this into consideration.

Below will follow specifications for all the instruction types.

\subsection{Instruction types}
Every instruction will take exactly one cycle. Almost every instruction needs
only one memory cell and should increment the \pc \  by one.
Any operation using  a argument i.e \A \ will occupy two memory cells and
increment the \pc \ by two.\\
Using jump operations may affect the \pc in other ways.
No operation except moves to the IRQ registers (\q \ and \qq ) are allowed.
If any other operation where to try to access the IRQ registers the opcode is
invalid and the VM should crash.

\subsubsection{Move operations}
The only invalid move operations are those where the second digit is a 8 since
one can not write to \A.
Although some are nonsensical such as \V+000011+ since it would move \y \ to \y \ .

\subsubsection{Arithmetic operations}
The increment($++$) and decrement($--$) operations only take one write argument
and the read argument should be ignored. Incrementing or decrementing a register
or memory cell updates the value stored in that registry directly and does not
affect any thing else.\\
The other arithmetic operations takes the write digit as the first argument to
the operation and the write operation will be the second argument.
The result of the operation is always stored on the stack.\\
If one tries division by zero a exception should be thrown and the VM shall
crash.

\subsubsection{Logic operations}
The logic operations work in the same way as the arithmetic operations. The
comparison operations will return 0 if the result is false and 1 otherwise.
Any logic operation where the 3:d digit is non zero is an illegal instruction
and a exception should be thrown and the VM shall crash.

\subsubsection{Jump operations}
The standard address jump (J) will jump the \pc \ to the address given by its
read digit.\\
The conditional breaks takes the write digit as its first argument and the read
digit as its second argument. If the test fails the \pc \ will skipp the next
instruction. This will require som tricks to implement. The VM must , at
runtime, identify weather or not the following instruction takes up one or two
memmory cells.\\

A $J_{SR}$ (subroutine jump) will take a argument in \A \ and move the \pc \
there and it will also put its current value on \J.\\
A return jump will jump to the address at the top of \J \ and then pop the
stack.If the jump where to be empty the VM should crash and an exception
should be raised.\\


\subsubsection{Special}
TThe Halt operation which just stops the VM and
raises an exception.//
And the \V+SEM+ or Stack empty operation which returns 1 if the stack is empty
and 0 else.
The \V+POP+ just pops the stack. I.e removing the top object.

\subsection{Opcodes}

\begin{tabular}{l || c | *{7}{c|} | r}
\textbf{Mnemonic} & \textbf{Description} & \x & \y& \s & \A &\ma & \mx & \my &
Args
\\
\hline
\V+NOP+ & No Operation & x & x & x & x & x & x & x & 0  \\
\hline
\V+MOV+ & Move operations & b & b & b & r & b & b & b & 1 \\
\hline
\V+INC+ & Increment& b & b & x & x & x & x & x & 1 \\
\V+DEC+ & Decrement	& b & b & x & x & x & x & x & 1 \\
\V+ADD+ & Add		& r & r & w & r & r & r & r & 2 \\
\V+SUB+ & Subtract	& r & r & w & r & r & r & r & 2 \\
\V+MUL+ & Multiply	& r & r & w & r & r & r & r & 2 \\
\V+DIV+ & Division	& r & r & w & r & r & r & r & 2 \\
\V+MOD+ & Modulus	& r & r & w & r & r & r & r & 2 \\
\hline
\V+EQL+ & Equal		& r & r & w & r & r & r & r & 2 \\
\V+LES+ & Less		& r & r & w & r & r & r & r & 2 \\
\V+GRT+ & Greater 	& r & r & w & r & r & r & r & 2 \\
\V+BRL+ & Rotate L	& r & r & w & r & r & r & r & 1 \\
\V+BRR+ & Rotate R	& r & r & w & r & r & r & r & 1 \\
\V+AND+ & And		& r & r & w & r & r & r & r & 2 \\
\V+ORR+ & Or 		& r & r & w & r & r & r & r & 2 \\
\V+XOR+ & Xor		& r & r & w & r & r & r & r & 2 \\
\V+NOT+ & Not		& r & r & w & r & r & r & r & 2 \\
\hline
\V+JMP+ & Jump		& r & r & x & r & r & r & r & 1 \\
\V+BEQ+ & Jump Equal& r & r & r & r & r & r & r & 2 \\
\V+BLE+ & Jump Less	& r & r & r & r & r & r & r & 2 \\
\V+BGR+ & Jump Greater& r & r & r & r & r & r & r & 2 \\
\V+JSR+ & Subroutine Jump& r & r & x & r & r & r & r & 1 \\
\V+RET+ & Return Jump& x & x & x & x & x & x & x & 0 \\
\hline
\V+HLT+ & HALT& x & x & x & x & x & x & x & 0 \\
\V+SEM+ & Stack Empty& x & x & w & x & x & x & x & 0 \\
\V+POP+ & Pop Stack& x & x & w & x & x & x & x & 0 \\
\end{tabular}
 

\end{document}